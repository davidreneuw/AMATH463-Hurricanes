\chapter{Introduction}

With the growing threat of climate change becoming more of an emphasis in research and policy, the study of weather phenomena has become a larger focus of physical sciences. Ocean flows and air flows in our atmosphere have life-changing impacts on people's lives around the world, and being able to predict such phenomena is part of the role science must play on society. As mentioned by \cite{noaa}, with an average of 17.2 average annual weather and climate disasters in the U.S. alone, a total of above \$2 trillion in damages has been caused by hurricanes since 1980, with \$20 billion in 2021 alone. There is more and more merit to understanding these phenomena in order to diminish their impact and potentially even avert them.

Meteorology is a science that dates back thousands of years: civilizations as early as Babylonians recorded weather observations, attempting to explain the actions of the atmosphere as well predicting winds and rains. However, by nature of the problem, making accurate predictions was nearly impossible as there was no way to take in account the nearly infinite parameters while retaining sufficient precision in order to make any non-negligible time predictions. \cite{nebeker1995calculating}

% something about how the world wars have had an enormous impact on computing
Before the first World War, scientist Lewis Fry Richardson had derived an arithmetic method to solving partial differential equations and had in mind to apply it in Vilhelm Bjerknes' calculational approach to weather forecasting. Although Richardson's computational approach took 6 weeks to calculate a 6 hour advance in weather, thus discouraging others from the computational approach, major developments were made between the World Wars (cold/warm fronts, air masses and polar fronts). After the second war, John von Neumann wanted to demonstrate the potential of computers. By 1956, he had succeeded by computationally forecasting weather as accurately and reliably as human meteorologists but much faster than them. \cite{nebeker1995calculating}

% something about modern meteorology // more specifically recent history about atmosphere science
Since then, computers have become the norm in meteorology and atmospheric sciences. With all the advances in technology, the problem has gotten back to a limitation in theory, especially in the genesis of hurricanes: how hurricanes come to life is still not fully understood. We have a great understanding of the conditions required for hurricanes to form but these conditions do not guarantee a hurricane to form: what decides if a hurricane is born or not from met conditions is a mystery \cite{develop}.

This, however, does not mean that we cannot glean important information about hurricanes from examining specific aspects. This report will explore a number of concepts relating to hurricanes. First we discuss the necessary fluid mechanical background, including the Coriolis force and geostrophic flow. Then we briefly discuss the conditions for genesis of a hurricane, before deriving the governing equations of mature hurricanes. We then take a numerical approach to finding the temperature profile for a hurricane from the tangential wind speed. 

\begin{center}
    \includegraphics[width=0.8\linewidth]{assets/hurricane.png}
\end{center}