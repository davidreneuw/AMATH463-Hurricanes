\chapter{Conclusion}
\section{Discussion}
Hurricane formation is an ever changing topic with many of its mechanism still not fully explained: for example, how the eye of the storm is created and maintained, how vertical wind shears inhibits hurricane formation and what decides whether a tropical depression with the right conditions becomes a hurricane or not are all unclear subjects even today.

Despite the complexity of the subject matter, we were able to identify many of the mechanisms that drive the formation of hurricanes: from fluid mechanics to thermodynamics. Topics of the latter were not deeply investigated as they were not part of the scope of this course and project.

\section{Next steps}
With modern meteorology's focus on computational methods, it would have been relevant to include novel computational and numerical plots. At least, having individual mechanics of hurricane formation computationally demonstrated would have been a great experience.

A deeper look into the thermodynamics behind secondary circulation in hurricanes would have provided a more complete look at the topic at hand.

\section{Acknowledgement}
We would like to give Dr. Lamb who gave the AMATH463:\textit{Fluid Dynamics} course many thanks for his patience and help throughout this project and this term. We would also like to thank our teaching assistant for their great work this term.