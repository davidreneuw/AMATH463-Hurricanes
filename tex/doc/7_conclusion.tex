\chapter{Conclusion}
\section{Discussion}
Hurricane formation is an ever changing topic with many of its mechanism still not fully explained: for example, how the eye of the storm is created and maintained, how vertical wind shears inhibits hurricane formation and what decides whether a tropical depression with the right conditions becomes a hurricane or not are all unclear subjects even today.

Despite the complexity of the subject matter, we were able to identify many of the mechanisms that drive the formation of hurricanes: from fluid mechanics to thermodynamics. Topics of the latter were not deeply investigated as they were not part of the scope of this course and project, however we were able to come up with a simple toy model for the temperature profile in a hurricane with the knowledge of its tangential wind speed. 

\section{Next steps}
As mentioned, the dynamics of hurricanes are extremely complex. We saw in our temperature profile that our temperatures were non-physical for radii near the eye of the hurricane. Further investigation into the dynamics of the eyewall and convection at the centre of the storm would be an interesting theoretical and computational exercise. Works such as \cite{smith_montgomery_2017} and \cite{Ooyama1969} would provide a good basis for this exercise.

\section{Acknowledgement}
We would like to give Dr. Lamb who gave the AMATH463:\textit{Fluid Dynamics} course many thanks for his patience and help throughout this project and this term. We would also like to thank our teaching assistant for their great work this term.