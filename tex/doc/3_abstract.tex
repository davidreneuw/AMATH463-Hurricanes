\chapter{Abstract}
\begin{tcolorbox}[title=Abstract,colbacktitle=white!80!black,coltitle=black,arc=0mm,boxrule=0.1mm]
Hurricane formation is an impactful topic that is not yet fully understood but we identified many relevant fluid dynamics and thermodynamics mechanisms that play important roles in their creation. Cold fronts above warm water cause evaporation, which upon receiving low vertical wind shear and happening in moist atmosphere areas cause tropical depressions. Given large enough latitude for significant Coriolis force but not too large to have high enough surface temperature, tropical depressions can accumulate energy and gain vorticity through geostrophic flow, at which point they become tropical storms. Tropical storms intensify through drift and conservation of angular momentum, until they have winds strong enough to be called a hurricane. Once a hurricane reaches land, their energy supply (latent heat dissipation from evaporation of sea water) is cut off and they "rapidly" die. This report discusses hurricanes from their genesis to maturity, relating their dynamics to fluid mechanical concepts and applying a numerical approach to finding temperature profiles through the storm.
\vspace{5mm}

\textbf{Key terms:} Geostrophic flow, evaporation and convection, conservation of angular momentum, rotating frame of reference
\end{tcolorbox}